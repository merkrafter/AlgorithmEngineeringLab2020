% arara: lualatex
% arara: bibtex
% arara: lualatex
% arara: lualatex
% arara: clean: {files:[paper.aux, paper.bbl, paper.blg, paper.log, paper.out]}

\documentclass[sigconf]{acmart}
\usepackage[english]{babel}
\usepackage{csquotes}
\usepackage{booktabs}
\usepackage{tikz}
\usetikzlibrary{matrix.skeleton}

\newcommand{\bl}{\node [circle, minimum size=0.7cm, draw=black, fill=blue!65!white, thin]{};}
\newcommand{\wh}{\node [rectangle, minimum size=0.7cm, draw=black, fill=yellow] {};}


%% \BibTeX command to typeset BibTeX logo in the docs
\AtBeginDocument{%
  \providecommand\BibTeX{{%
    \normalfont B\kern-0.5em{\scshape i\kern-0.25em b}\kern-0.8em\TeX}}}

%% These commands are for a PROCEEDINGS abstract or paper.
\settopmatter{printacmref=false} % Removes citation information below abstract
\renewcommand\footnotetextcopyrightpermission[1]{} % removes footnote with conference information in 

\acmConference[AELAB 2021]{AELAB 2021: Algorithm Engineering LAB Projects}{March 1}{Jena, Germany}

% convert text to title case
% http://individed.com/code/to-title-case/

% that helps you to formulate your sentences
% https://www.deepl.com/translator

\begin{document}

\title[San Jego]{Efficient Bot for The Game of San Jego\\\large Algorithm Engineering LAB 2021 Project Paper}

\author{Mark Umnus}
\affiliation{%
  \institution{Friedrich Schiller University Jena}
  \country{Germany}}
\email{mark.umnus@uni-jena.de}

%% The abstract is a short summary of the work to be presented in the article.
\begin{abstract}

\begin{enumerate}
\item \textbf{Topic and background:} What topic does the paper deal with? What is the point of departure for your research? Why are you studying this now?
\item \textbf{Focus:} What is your research question? What are you studying precisely?
\item \textbf{Method:} What did you do?
\item \textbf{Key findings:} What did you discover?
\item \textbf{Conclusions or implications:} What do these findings mean? What broader issues do they speak to?
\end{enumerate}


\end{abstract}

\keywords{combinatorial game theory, optimization, bot}

\maketitle

\let\thefootnote\relax\footnotetext{AELAB 2021, March 1, Jena, Germany. Copyright \copyright 2021 for this paper by its authors. Use permitted under Creative Commons License Attribution 4.0 International (CC BY 4.0).}


\section{Introduction}

San Jego is a deterministic board game for two players (blue and yellow) that both have perfect information.
At the beginning of the game, the board (of arbitrary size) is filled with bricks in the colors of the players in a checkerboard-like pattern.
From now on, these bricks will be treated as \emph{towers of height 1}.
The players move alternatingly by choosing a \emph{source} tower that has a top brick in the color of the active player, and placing it onto an adjacent \emph{target} tower.
As a result, the source field is now empty and the target field contains a tower whose height is equal to the sum of the heights of the source and target tower and whose top brick still has a color equal to the player that just made the move.
A player must \emph{skip}, if there is no legal move as outlined above.
In that case, the other player may make moves as long as legally possible.
After that, the game is over with the winner being the player with the highest tower on the board.


\subsection{Background}

\subsection{Related Work}
San Jego must be viewed in the broader context of board game research.
Engines for these games typically perform three kinds of actions: move generation, search and evaluation \cite{Bimonugroho2020}.
% TODO more on that

For instance in chess, relevant features of the board are commonly represented as bitboards where each field corresponds to one bit in a 64 bit integer \cite{Bimonugroho2020}.
One such bitboard may track the fields that have black pieces on them, and another one may track the positions of knights.
Finding all black knights on the board is now as efficient as computing a bitwise \texttt{AND} of these two bitboards.

This particular game has not received much attention yet.
Major contributions were made by \citeauthor{Althöfer2020} \cite{Althöfer2020}.
They proved an upper bound for the game state complexity, the branching behavior, and precise values for games on small boards.
The optimal move lines that lead to these values were not published, though.

\subsection{Our Contributions}
\begin{itemize}
  \item theoretical analysis
  \item Framework:
  \begin{itemize}
    \item San Jego library
    \item reference bot implementation
    \item metrics
  \end{itemize}
\end{itemize}

\subsection{Outline}
The following section discusses key properties of the game, including ones that can be exploited for an efficient implementation, and those that restrict common optimizations in one way or another.

\section{Theoretical Analysis}
This section is subdivided into two parts.
First, some low-level features of the game are discussed, and it is investigated how they can be used to increase the efficiency of San Jego search programs.
After that, it is shown how these individual parts fit into the greater picture of well-known algorithms for combinatorial games.

\subsection{Low-level considerations}\label{subsec:low-level}
\paragraph{Tower design}
This first paragraph focuses on the central objects of this game: the towers.
Towers are built from the bricks the game starts with by stacking them onto another, meaning that the maximum height they can reach is bounded by the board size.
Important operations on them include the retrieval of height and color of the top brick, as the latter identifies the owner of the tower, and obviously an operations to place them on top of each other.
Moreover, as we will see later, it is beneficial to offer a way to \enquote{subtract} towers from each other, that is revert the stacking.

The naive implementation would use a growable, list-like structure of integers to store the bricks.
This would allow constant access times to height and top brick, and linear time complexity to merge two towers.
As the bricks can only be in one of two states, namely blue and yellow, this would waste a lot of memory.
The next logical step is to store only a single bit for each brick using an appropriate integer type.
It is not practical, however, to just assign 0 and 1 to the available colors, as leading zeros in an integer would cause a loss of information.
Hence, the mapping must be normalized so that the top brick is always represented by a 1, and the actual color must be encoded separately.
Using this representation, all operations can be carried out in constant time using bit manipulations such as the usual \texttt{OR}, shifts and (masked) \texttt{XOR} as well as the advanced \texttt{LZCNT} (leading zeros count) which are mapped to efficient instructions on a broad variety of architectures.
A last modification is possible if the actual structure of bricks in a tower does not matter as it is the case for the standard ruleset considered in this paper.
Then, the integer can be used to store the height value directly, reducing the space complexity logarithmically.
Adding two towers now simplifies to adding their heights and setting the top brick correctly.

\paragraph{Symmetry}
As already mentioned, the game can be played on rectangular boards of arbitrary sizes.
This makes it difficult to apply size-specific optimizations like bitboards.
However, independent of its shape, the game board is symmetrical along both axes.
This means that on a $3\times3$ board, say, moving the center tower up or down in the first turn yields equivalent game states.
In addition, square boards are rotationally symmetrical, that is on this particular board, moving the center tower left or right also yields states equivalent to the ones mentioned before.
So far, openings have not been studied deeply, and no best first move has been discovered for arbitrary board shapes.
Therefore, exploiting the symmetries is the main tool for reducing the number of moves to consider in the first turns.
They allow to ignore roughly half the moves on average.
On square boards and boards with only odd side lengths, another factor of ca. $0.5$ can be ignored, respectively.
The proof and exact numbers are not provided here, however, as they are too verbose for this paper.
Moreover, by exploiting symmetries it is possible to build and store tables of interesting positions more effectively.

\paragraph{Isolation}
The game rules state that towers can only be placed on top of other towers.
In particular, they must not be moved to fields that are already empty.
Several consequences follow from this.
First, the number of towers is an upper bound for the overall number of moves in a game as with each move the number of towers on the board must decrease by one (the number of towers $-1$ to be precise, as there is no legal move with only one tower on the board).
Although the number of moves in a game is usually somewhat lower (on average 72\% of the maximum, although the number is higher if both players are good \cite{Althöfer2020}) this can possibly be used for time management to estimate the remaining length of a game.
Second, if a player at some point has no legal move left, they have to skip.
In this situation it is clear that none of their towers has an adjacent tower, otherwise skipping is not allowed.
It follows that also in the future progression of the game, none of the skipping player's towers will be able to move.
At this point, the evaluation module does not have to search a game tree anymore, but only count the maximum height reachable for the player that is still active.
Third, as the game progresses the board gets sparse.
Analogously to matrices, it might be beneficial to use a representation different from an array in that case.
While this likely plays a minor part for smaller boards, the effects on larger boards can be dramatic, especially in the end game where large portions of the board are free.
During move generation, all towers of the active player have to be iterated.
This operation is highly inefficient if, for instance, 70\% of all fields are empty.
Here, using a map to store the board (or possibly two maps: one for each player) should reduce the runtime considerably.
Towers that have no neighbors can easily be ignored during move generation as they will not be able to move ever again.
However, move generation is not the only computational cost.
The other part is the evaluation step.
For that, \emph{all} towers on the board must be iterated to consider their height.
Early on, this process is perfectly efficient using an array representation, allowing vectorized processing.
Small boards, moreover, fit completely into a single cache line in that model.
Altogether, it is expected that there is some turn number (dependent on the board \emph{shape}) at which it is beneficial to change the board representation from an array to a map.
(The word \emph{shape} is used instead of \emph{size} to emphasize that computations on a $1\times15$ board are much easier than on a $3\times5$ board, although the number of fields is equal \cite{Althöfer2020}.)
As this number is probably also dependent on the hardware being used (e.g. the vector instruction set), it is suggested to leave this up to the user's choice via a parameter.
The evaluation step also carries room for minor optimizations in a map representation similar to the ones mentioned above at the generation phase.
Isolated towers may be immovable, but they are also \emph{safe}:
the opponent can not take them over, so they form an upper bound for the margin a player can loose by.
This means that only a single isolated tower with the maximum height for each player has to be stored as all others won't affect the outcome.
The towers on the board tend to decay into many isolated regions in the end game.
This behavior is well known from the related game Clobber \cite{Althöfer2004}.
Therefore, ignoring a large portion of irrelevant towers in the end game allows the evaluation module to find the truly important towers in the current state.
This technique only works for single isolated towers, though.
It is not possible to ignore whole regions of connected towers whose sum of heights is lower than the height of some other isolated tower.
As soon as towers are connected, they can be moves which means that they contribute to maneuvers like zugzwang in other regions on the board.
This can be ignored, of course, if some player has a safe tower that is higher than any other active region can sum up to, but this happens fairly rarely if both players are about equally strong.

% TODO find a good headline
\paragraph{Board value caching}
The next optimization is based on observing when the game value changes.
As it is only based on the highest towers, it is inherently robust to most changes on the board.
Hence, much computation can be saved when the height of the highest towers for each player is cached.
It only has to be updated in two cases:
First, a player creates a new highest tower.
In that case, the cached number just has to be overwritten which is a cheap operation.
Second, the opponent takes over a tower with the same height as the cached maximum.
This is the only (comparatively) expensive case as the whole board must be searched to find the new highest tower.
The cache can be extended to the top $k$ towers for each player to delay the full search case.
It is open for which $k$ the cache management becomes too costly to be effective (remember that the first case creates linear overhead for $k$ cached heights).

\subsection{High-level considerations}
\paragraph{Search algorithm}
Searching game trees typically comes in two flavors:
\enquote{full} search and Monte Carlo search (or a combination thereof).
The former searches every possible game state that is reachable from the current state, and returns a path that leads to the best outcome for the active player.
It is most effective if the game state complexity is manageable.
Otherwise, this method may not find the best (or even a good) move within acceptable time.
It is required to use this method, if exact game values are needed, e.g. in theoretical analysis of the game.
In practical settings, some form of $\alpha$-$\beta$-pruning is always applied as it cuts off branches of the search tree that will not influence the outcome.
For the pruning to work most effectively, the best move has to be searched first.
This leads to a chicken-and-egg problem that is typically solved by using a heuristic for the move ordering as discussed in the next paragraph.

Monte Carlo tree search (MCTS) is applied in settings where there are too many states to search completely (or even pruned).
Its core idea is to consider random moves, collect statistics about their outcomes, and choose the move that has the highest probability of being good.

In San Jego, the game state space becomes unmanageable quite early.
\citeauthor{Althöfer2020} report that the full search on boards as small as $5\times5$ with their NegaScout implementation (an optimized version of $\alpha$-$\beta$-pruning) took \enquote{several days} \cite[p.~8]{Althöfer2020}.
Using the optimizations outlined in section \ref{subsec:low-level}, it is likely possible to bring that number down.
Typical computer olympiad matches of similar games, however, are played on $10\times10$ boards, which is completely hopeless to encounter with full search.
Therefore, MCTS should be used for San Jego playing programs.
One important property here is that the branching factor monotonically decreases in the course of the game, as each turn a tower is effectively eliminated.
This makes full tree search applicable in the end game to play that phase more precisely.

\paragraph{Search strategy}
An important factor for the success of game tree searching is the evaluation function.
In some combinatorial games like chess and Go, the value of a game state is generally not known.
Other games, as San Jego, define a value for each game state, but it might not be good approach for sorting candidate moves.
For instance, consider the first turn in a game.
Whatever move the player chooses, they all lead to a game state with the same value of $+1$.
Therefore, several improvements are possible.
\citeauthor{Althöfer2020} found that a combination of prioritizing moves that target high enemy towers and deprioritizing moves that target high own towers on the one hand, and prioritizing moves targetting the center of the board on the other hand work together nicely.


%\begin{center}
%\begin{tikzpicture}
%\matrix (m) [matrix of nodes, style grid={draw}, nodes in empty cells, label skeleton, row sep=2mm, column sep=2mm, nodes={minimum size = 0.8cm}] {
% \bl & \bl & \bl  \\ % for some reason this only works with an empty cell
% \wh &     & \wh  \\ 
% \bl & \wh & \bl  \\
%};
%\end{tikzpicture}
%\end{center}

\begin{itemize}
  %\item board is symmetrical and rotationally invariant -> speed up start
  %\item towers can not move to empty fields -> each turn a tower is removed -> max number of turns is fixed and depends on board size
  %\item what move to choose in first turn? all moves produce same board value -> empirically: moves to center of boards are more powerful on average \cite{Althöfer2020}
  %\item max tower height: m*n => can be used for representation improvement
  %\item cache the hights of the highest towers of both players (avoids costly recalculation in every explored node); recalculation only necessary if new higher tower was built or a tower with the max height of a player was taken
  %\item if a player had to skip once, there will never be a move available to them again
  %\item isolated regions
\end{itemize}

\section{The Algorithm}
\begin{itemize}
  \item tower representation
  \item fit board in cache line and align
  \item use vector SIMD to compute board value
  \item use map instead of array for sparse boards in end game
  \item use opening table
  \item traverse game tree in parallel
\end{itemize}

\subsection{Internal Representation of Mock Labels}
\label{sub:sec:internal}

In Figure~\ref{fig:integer:sets} we convert the mock labels to sorted integer sets.



\begin{figure}[htbp]
  \centering
  \includegraphics[width=\linewidth]{./graphics/integer_sets.pdf}
  \caption{Conversion of mock camera labels to sorted integer sets. 
We map each unique token (key) in camera labels to a unique value. 
Based on these key-value-mappings, we convert camera labels to sorted integer sets. 
A camera can have different names in different countries. Therefore, repeating IDs reference the same cameras (see, for example, ID=3).} 
  \label{fig:integer:sets}
\end{figure}

\subsection{Efficient Preprocessing of Input Data}
\label{sub:sec:preprocessing}

The following findings are important to speed up preprocessing of the input data:

\begin{itemize}
\item Reading many small files concurrently, with multiple threads (compared to a single thread), takes advantage of the internal parallelism of SSDs and thus leads to higher throughput \cite{Zhuang2016}.

\item C-string manipulation functions are often significantly faster than their C++ pendants. For example, locating substrings with \texttt{strstr} is around five times faster than using the C++ \texttt{std::string} function \texttt{find}.

\item Hardcoding regular expressions with \emph{while, for, switch} or \emph{if-else} statements results in faster execution times than using standard RegEx libraries, where regular expressions are compiled at runtime into state machines.

\item Changing strings in place, instead of treating them as immutable objects, eliminates allocation and copying overhead.

\end{itemize}


\section{Experiments}

Table~\ref{tab:results} shows the running times of the resolution step of the five best placed teams.


\begin{table}[htbp]
  \caption{Comparison of the F-measure and the running times of the resolution step of the five best placed teams. The input data for the resolution step consisted of 29{,}787 in JSON formatted e-commerce websites. Measurements were taken on a
laptop running Ubuntu 19.04 with 16 GB of RAM and two Intel Core i5-4310U CPUs. The underlying SSD was a 500\,GB 860 EVO mSATA. We cleared the page cache, dentries, and inodes before each run to avoid reading the input data from RAM instead of the SSD.}
  \label{tab:results}
\resizebox{\columnwidth}{!}{
  \begin{tabular}{lcrr}
    \toprule
    Team& Language & F-measure & Running time (s)\\
    \midrule
	PictureMe (\textbf{this paper}) &C++& 0.99 & \textbf{0.61}\\
    DBGroup@UniMoRe &Python& 0.99 & 10.65\\
    DBGroup@SUSTech &C++& 0.99 & 22.13\\
    eats\_shoots\_and\_leaves &Python& 0.99 & 28.66\\
    DBTHU &Python& 0.99& 63.21\\
  \bottomrule
\end{tabular}
}
\end{table}


\section{Conclusions}

\bibliographystyle{ACM-Reference-Format}
\bibliography{literature}


\end{document}
\endinput
